% !TeX spellcheck = en_US
\documentclass[prb,superscriptaddress,letter,10pt,onecolumn]{revtex4}
\usepackage{amsmath}
\usepackage{amssymb}
\usepackage{graphicx}
\usepackage{color}
\usepackage{epsfig}
\usepackage{dcolumn}
\usepackage{bm}
\usepackage{bm}
\usepackage{blindtext}
\usepackage{natbib}
\usepackage{epstopdf}
\usepackage{ulem}
\usepackage{url}
\usepackage{bm}
\usepackage{subfigure}


\def\be{\begin{equation}}
\def\ee{\end{equation}}
\def\bea{\begin{eqnarray}}
\def\eea{\end{eqnarray}}
\def\erf{\eqref}
\def\ba{\begin{aligned}}
\def\ea{\end{aligned}}
\def\erf{\eqref}
\def\arccosh{\text{arccosh}}
\def\d{\text{d}}

\newcommand{\comment}[1]{{\color{red}\textit{#1}}}
\newcommand{\cancel}[1]{{\color{blue}\sout{#1}}}
\newcommand{\msout}[1]{\text{\sout{\ensuremath{#1}}}}
\newcommand{\expct}[1]{\left\langle #1 \right\rangle}
\newcommand{\expcts}[1]{\langle #1 \rangle}
\newcommand{\ud}          {\mathrm d}
\newcommand\eps           {\varepsilon}
\newcommand\w           {\omega}
\newcommand\fii           {\varphi}
\newcommand\mc            {\mathcal}
\newcommand\LL            {Lieb--Liniger }
\newcommand\p             {\partial}
\newcommand\psid          {\psi^{\dagger}}
\renewcommand\th          {\theta}
\newcommand\kb            {k_\text{B}}
\newcommand \rhop       {\rho^{\text{(r)}}}
\newcommand{\cev}[1]   {\langle#1|}
\newcommand\red[1]{{\color{red}{#1}}}
\newcommand\blue[1]{{\color{blue}{#1}}}
\newcommand{\vt}{\vartheta}
\newcommand{\vev}[1]{\left\langle #1 \right\rangle}

\begin{document}
\title{Proof of a Possible Form of Green Function on a lattice system}
\author{Shuohang Wu}
\affiliation{Wilczek Quantum Center, School of Physics \& Astronomy, Shanghai Jiao Tong University, Shanghai, 200240, China, 020072910056}

\maketitle

\section{Statement}

For a lattice system of any dimension with total size (number of lattice point) $L$, define annihilation operator as $C_i$ and creation operator as $C_i^\dagger$ for the i-th point.
For a single particle, there are $L$ independet eigenstate $\left|E_k\right>$, with corresponding eigenenergy $E_k$.
We can write
\begin{equation}
	\left|E_k\right> = U^{-1}\left|i\right>,
	\left|i\right> = U\left|E_k\right>,
\end{equation}
where $\left|n\right>$ denotes the occupied state at n-th point.
Assume the unitary transformation matrix $U$ takes the form
\begin{equation}
	U =       	
	\left(                 	
	\begin{array}{cccc}   		
		a_{k_11} & a_{k_21} & \cdots & a_{k_L1}\\  		
		a_{k_12} & a_{k_22} & \cdots & a_{k_L2}\\
		\vdots & \ddots &  & \\
		a_{k_1L} & a_{k_2L} & \cdots & a_{k_LL}\\  		
	\end{array}	
	\right),         	
\end{equation}
and
\begin{equation}
	\left|E_k\right> = C_k^\dagger\left|0\right>,
	\left|i\right> = C_i^\dagger\left|0\right>,
\end{equation}
then the transformations of the annihilation operator and creation operator are
\begin{equation}
	C_i = \sum_{k}a_{ki} C_k,
	C_i^\dagger = \sum_{k}a_{ki}^* C_k^\dagger
\end{equation}

Generally, the Green function for the correlation between the i-th and j-th point can be written as
\begin{equation}
	G_{ij} = \left< C_i^\dagger C_j \right>.
\end{equation}
We can PROVE that for an N-particle ground states ($N\le L$), a possible form of the Green function is
\begin{equation}
	G = 
	\left(                 	
	\begin{array}{cccc}   		
		a_{k_11}^* & a_{k_21}^* & \cdots & a_{k_N1}^*\\  		
		a_{k_12}^* & a_{k_22}^* & \cdots & a_{k_N2}^*\\
		\vdots & \ddots &  & \\
		a_{k_1L}^* & a_{k_2L}^* & \cdots & a_{k_NL}^*\\  		
	\end{array}	
	\right)_{L\times N}
	\left( 
	\begin{array}{cccc}   		
		a_{k_11} & a_{k_12} & \cdots & a_{k_1L}\\  		
		a_{k_21} & a_{k_22} & \cdots & a_{k_2L}\\
		\vdots & \ddots &  & \\
		a_{k_N1} & a_{k_N2} & \cdots & a_{k_NL}\\  		
	\end{array}	
	\right)_{N\times L}.
\end{equation}

\section{Lemma}

To prove the statement above, we introduce the Wick's theorem as follow.
\begin{equation}
	\left< C_1C_2\dots C_{2N-1}C_{2N} \right> = \sum_{P} sgn(P)\left< C_iC_j\right>\left< C_kC_l\right>\cdots\left< C_mC_n\right>,
\end{equation}
where $N$ is an integer and the right-hand-side of the equation takes the sum of all combinations of expetation value for two operators. $\{i,j,k,l,\cdots,m,n\}$ is any permutation of $\{1,2,\cdots,2N-1,2N\}$, and $sgn(P)$ denotes the sign of the permutation, while the relative position bewteen any annihilation and creation operator pair should be preserved.
Here the operators can be annihilation or creation.
Obviously for a non-zero result the left-hand-side should be both of number $N$ and any $\left< C_iC_j\right>$ should contains one annihilation and one creation operator.

For example, if the state is a vaccum state and $N=2$, the Wick's theorem shows that
\begin{equation}
	\left< C_1 C_2 C_3^\dagger C_4^\dagger \right> =
	-\left< C_1 C_3^\dagger\right>\left< C_2 C_4^\dagger\right>
	+\left< C_1 C_4^\dagger\right>\left< C_2 C_3^\dagger\right>, 
	\left< C_1 C_2^\dagger C_3 C_4^\dagger \right> =
	\left< C_1 C_2^\dagger\right>\left< C_3 C_4^\dagger\right>.
\end{equation}

\section{Proof}

The $N=1$ case can be easily proved using relations of annihilation and creation operators, without using Wick's Theorem.
Now we prove the case for $N=2$.
A two-particle eigenstate can be written as
\begin{equation}
	\left|E\right> = C_{k_1}^\dagger C_{k_2}^\dagger\left|0\right>.
\end{equation}
Then the Green function can be wriiten as
\begin{equation}
	G_{ij} = 
	\left< 0 \left| C_{k_2}C_{k_1}  C_i^\dagger C_j
	C_{k_1}^\dagger C_{k_2}^\dagger\right|0\right>.
\end{equation}
After performing unitary transformation on $C_i^\dagger$ and $C_j$, that is
\begin{equation}
	C_j = \sum_{k^{'}}a_{k^{'}j} C_{k^{'}},
	C_i^\dagger = \sum_{k}a_{ki}^* C_k^\dagger,
\end{equation}
the Green function becomes
\begin{equation}
	G_{ij} = \sum_{kk^{'}} a_{ki}^* a_{k^{'}j}
	\left< 0 \left| C_{k_2}C_{k_1}  C_k^\dagger C_{k^{'}}
	C_{k_1}^\dagger C_{k_2}^\dagger \right|0\right>.
\end{equation}
Use Wick's theorem and notice the fact that
\begin{equation}
	\left< 0 \left| C_m C_n^\dagger \right| 0 \right> = \delta_{mn},
\end{equation}
which leads to two zero terms, so
\begin{equation}
	\begin{aligned}
	&\left< 0 \right| C_{k_2}C_{k_1}  C_k^\dagger C_{k^{'}}
	C_{k_1}^\dagger C_{k_2}^\dagger\left|0\right> \\
	= &\left< 0 \right| C_{k_2} C_k^\dagger \left| 0 \right>
	\left< 0 \right| C_{k_1} C_{k_1}^\dagger \left| 0 \right>
	\left< 0 \right| C_{k^{'}} C_{k_2}^\dagger \left| 0 \right> \\
	&+ \left< 0 \right| C_{k_2} C_{k_2}^\dagger \left| 0 \right> 
	\left< 0 \right| C_{k_1} C_{k}^\dagger \left| 0 \right>
	\left< 0 \right| C_{k^{'}} C_{k_1}^\dagger \left| 0 \right> \\
	= & \quad\delta_{k_2 k} \delta_{k^{'}k_2} + \delta_{k_1 k} \delta_{k^{'}k_1}.
	\end{aligned}		
\end{equation}
Thus the Green function becomes
\begin{equation}
	G_{ij} = a_{k_1i}^* a_{k_1j} + a_{k_2i}^* a_{k_2j},
\end{equation}
which can be also written as
\begin{equation}
	G = 
	\left(                 	
	\begin{array}{cc}   		
		a_{k_11}^* & a_{k_21}^* \\  		
		a_{k_12}^* & a_{k_22}^* \\
		\vdots & \vdots \\
		a_{k_1L}^* & a_{k_2L}^* \\  		
	\end{array}	
	\right)_{L\times 2}
	\left( 
	\begin{array}{cccc}   		
		a_{k_11} & a_{k_12} & \cdots & a_{k_1L}\\  		
		a_{k_21} & a_{k_22} & \cdots & a_{k_2L}\\	
	\end{array}	
	\right)_{2\times L}
	(QED.).
\end{equation}

For the cases of $N>2$, similar process can be followed.

%\newpage
%\bibliography{main}
%\bibliographystyle{apsrev-nourl}


%\newpage

%\appendix

\onecolumngrid
%\setcounter{figure}{0}
%\makeatletter
%\renewcommand{\thefigure}{S\@arabic\c@figure}
%\setcounter{equation}{0} \makeatletter
%\renewcommand \theequation{S\@arabic\c@equation}
%\section*{{ \Large Supplementary Material --- Spin dynamics of a perturbed quantum critical Ising chain with an $E_8$ symmetry}}


\end{document}
